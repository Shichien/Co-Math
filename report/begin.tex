\documentclass[14pt,a4paper,UTF8,twoside]{article}
% 用 documentclass 来声明文章的类型,一般用 article,可以百度搜搜还有什么可以选
% 12pt 一般是小四的大小,可以查查 pt 是什么单位
% 可以选择 a5、a6 等纸张
% UTF8 是编码格式,要用中文的话, UTF8 是最推荐的编码格式
% twoside 是打印的时候用双面打印,一般用不上

% Formatting Packages ——————————————————————————————————————
\usepackage{multicol} % 多行多列包
\usepackage{multirow}
\usepackage{enumitem} % 数字编号包
\usepackage{indentfirst}
\usepackage[toc]{multitoc} % 多行目录包

% Math & Physics Packages ————————————————————————————
\usepackage{amsmath, amsthm, amsfonts, amssymb} % 基础数学包
\usepackage{setspace}
\usepackage{physics}
\usepackage{cancel}
\usepackage{nicefrac}
\usepackage{unicode-math} % 允许数学公式使用特定字体
\usepackage{mdframed} % 注释包

% Image-related Packages —————————————————————————————
\usepackage{graphicx} % 插入图片要这个包
\usepackage{float} % 浮动体环境,用来调整图片的位置
\usepackage{subcaption} % 子图包
\usepackage{pgfgantt}
\usepackage{graphics, graphicx}
\usepackage{tikz, tikz-qtree}
\usetikzlibrary{arrows.meta, positioning, shapes}
\usetikzlibrary{shapes.geometric}
\tikzstyle{node_style} = [rectangle, rounded corners, draw, align=center, text width=3cm, minimum height=0.65cm]
\tikzstyle{arrow_style} = [thick, ->, >=stealth]

\usepackage{pgfplots}
\pgfplotsset{compat=1.18}
\usepackage{xcolor}
\usepackage{fourier-orns}
\usepackage{lipsum}

% 其余常用包 ————————————————————————————————————————————
\usepackage{booktabs} % 表格库
\usepackage{titlesec} % 标题库
\usepackage{fancyhdr} % 页眉页脚库
\usepackage[sorting=none]{biblatex}
\usepackage{array}

%—————————————页面基础设置———————————————%

\usepackage{geometry}
\geometry{left=10mm, right=10mm, top=20mm, bottom=20mm}
% 这个命令用来设置纸张上下左右的间距,可以自己调调试试

% 字体设置 ————————————————————————————————————————————————
\usepackage{fontspec} % 允许设置字体
\usepackage[utf8]{inputenc}
\usepackage{ctex}

% 代码块包 ————————————————————————————————————————————————
\usepackage{listings}

% ————————————————————————————————————————————————————————

% 导言区
% Colour Palette ——————————————————————————————————————
\definecolor{merah}{HTML}{F4564E}
\definecolor{merahtua}{HTML}{89313E}
\definecolor{biru}{HTML}{60BBE5}
\definecolor{birutua}{HTML}{412F66}
\definecolor{hijau}{HTML}{59CC78}
\definecolor{hijautua}{HTML}{366D5B}
\definecolor{kuning}{HTML}{FFD56B}
\definecolor{jingga}{HTML}{FBA15F}
\definecolor{ungu}{HTML}{8C5FBF}
\definecolor{lavender}{HTML}{CBA5E8}
\definecolor{merjamb}{HTML}{FFB6E0}
\definecolor{mygray}{HTML}{E6E6E6}
\definecolor{mygreen}{rgb}{0,0.6,0}
\definecolor{mymauve}{rgb}{0.58,0,0.82}

\lstset {
    backgroundcolor=\color{white},   % choose the background color; you must add \usepackage{color} or \usepackage{xcolor}
    basicstyle=\footnotesize,        % the size of the fonts that are used for the code
    breakatwhitespace=false,         % sets if automatic breaks should only happen at whitespace
    breaklines=true,                 % sets automatic line breaking
    captionpos=bl,                   % sets the caption-position to bottom
    commentstyle=\color{mygreen},    % comment style
    deletekeywords={...},            % if you want to delete keywords from the given language
    escapeinside={\%*}{*},           % if you want to add LaTeX within your code
    extendedchars=true,              % lets you use non-ASCII characters; for 8-bits encodings only, does not work with UTF-8
    frame=single,                    % adds a frame around the code
    keepspaces=true,                 % keeps spaces in text, useful for keeping indentation of code (possibly needs columns=flexible)
    keywordstyle=\color{blue},       % keyword style
    % language=Python,               % the language of the code
    morekeywords={*,...},            % if you want to add more keywords to the set
    numbers=left,                    % where to put the line-numbers; possible values are (none, left, right)
    numbersep=5pt,                   % how far the line-numbers are from the code
    numberstyle=\tiny\color{mygray}, % the style that is used for the line-numbers
    rulecolor=\color{black},         % if not set, the frame-color may be changed on line-breaks within not-black text (e.g. comments (green here))
    showspaces=false,                % show spaces everywhere adding particular underscores; it overrides 'showstringspaces'
    showstringspaces=false,          % underline spaces within strings only
    showtabs=false,                  % show tabs within strings adding particular underscores
    stepnumber=1,                    % the step between two line-numbers. If it's 1, each line will be numbered
    stringstyle=\color{orange},      % string literal style
    tabsize=2,                       % sets default tabsize to 2 spaces
    % title=Python Code              % show the filename of files included with \lstinputlisting; also try caption instead of title
}

\title{LateX 入门}

\begin{document}

\maketitle % 用来显示上面写的 \title{LateX 入门}

\section{LateX 是以环境为主的}

在 VSCode 中,Ctrl + Alt + B 是编译成 PDF 的快捷键.

\subsection{居中显示}

\begin{center}
    使用 center 来创建居中环境,注意,反斜杠符号是特殊符号,
    在 LateX 里是告诉编译器:我要开始一条命令了,所以直接输入反斜杠是不能显示的,
    要转义,转义的方法是输入三个斜杠 \\\
\end{center}

\textbf{文字加粗},\textit{文字斜体},\underline{文字下划线},代码用等宽字体:\texttt{This is some code}

% 在 figure 环境后添加 [H],这个是固定的,用来让这个图片进入浮动体环境,如果不添加这个 [H],图片可能不会按照你想要的位置排,所以每次插入图片都要加上 [H]

\begin{figure} [H]
    \centering % 图片居中的命令
    \includegraphics[width=0.35\textwidth]{../assets/img/1.png} % 可以自己调一下 width 的值试试
    \caption{示例图片标题}
    \label{fig:my_label} % 设置标签,以便后续引用
\end{figure}

要注意一下怎么对路径进行引用。像这个项目里,结构如下:

% 


tex 文件是你编辑的文件,要找到你的图片,向上返回一层后,到 report,再向上返回一层,到 project 的根目录,然后进入 assets,再进入 img,然后找到了 1.png,所以要用 ../assets/img/1.png

这里是引用图片的例子:如图 \ref{fig:my_label} 所示。 % 这里对图片进行引用

\subsection{子标题}

写点东西,再写点东西,再写点。

\subsubsection{子子标题}

\section{表格}

% 三线表
如表格 \ref{tab:my_label} 所示

\begin{table} [H]
    \centering
    \begin{tabular}{cccc} % columns 的数量要和下面的数据对应
        \toprule
        A & B & C & D \\
        1 & 2 & 3     \\
        \midrule
        4 & 5 & 6     \\
        7 & 8 & 9     \\
        \bottomrule
    \end{tabular}
    \caption{示例表格}
    \label{tab:my_label}
\end{table}

如表格 \ref{tab:my_label_2} 所示.

\begin{table} [H]
    \centering
    \begin{tabular}{cccc} % columns 的数量要和下面的数据对应
        \toprule
        A & B & C & D \\
        1 & 2 & 3     \\
        \midrule
        4 & 5 & 6     \\
        7 & 8 & 9     \\
        \bottomrule
    \end{tabular}
    \caption{示例表格 2}
    \label{tab:my_label_2}
\end{table}

\subsection{子图包}

\begin{figure} [H]
    \centering
    \begin{subfigure}{0.45\textwidth}
        \centering
        \includegraphics[width=0.9\textwidth]{../assets/img/1.png}
        \caption{子图 1}
        \label{fig:subfig1}
    \end{subfigure}

    \

    \begin{subfigure}{0.45\textwidth}
        \centering
        \includegraphics[width=0.9\textwidth]{../assets/img/1.png}
        \caption{子图 2}
        \label{fig:subfig2}
    \end{subfigure}
    \caption{示例子图}
    \label{fig:subfig}
\end{figure}

6666

\vspace{1cm}

7777

\newpage % \newpage{5} 表示分页,数字表示分页的数量

\subsection{分页}

% \newpage

8888

\subsection{分栏}

\begin{multicols}{2} % 开始两栏排版
    \section{基础介绍}
    \subsection{分栏优势}
    分栏排版常见于学术期刊和会议论文集,具有以下优点:

    % 无序列表
    \begin{itemize}
        \item 提高页面利用率
        \item 增强可读性
        \item 方便图表与文字对照
        \item 适应不同阅读场景
    \end{itemize}

    % 有序列表
    \begin{enumerate}
        \item 提高页面 $ x^2 + y^2 = z^2 $ 利用率
        \item 增强可读性
        \item 方便图表与文字对照
        \item 适应不同阅读场景
    \end{enumerate}

    \subsection{数学公式示例}
    洛伦兹方程:
    \begin{equation}
        \begin{cases}
            \frac{dx}{dt} = \sigma(y - x)   \\
            \frac{dy}{dt} = x(\rho - z) - y \\
            \frac{dz}{dt} = xy - \beta z
        \end{cases}
    \end{equation}

    麦克斯韦方程组:
    $$
        \begin{aligned}
            \nabla \cdot \mathbf{E}  & = \frac{\rho}{\varepsilon_0}                                                          \\
            \nabla \cdot \mathbf{B}  & = 0                                                                                   \\
            \nabla \times \mathbf{E} & = -\frac{\partial \mathbf{B}}{\partial t}                                             \\
            \nabla \times \mathbf{B} & = \mu_0\left(\mathbf{J} + \varepsilon_0 \frac{\partial \mathbf{E}}{\partial t}\right)
        \end{aligned}
    $$

    \section{详细内容}
    \lipsum[1-2] % 生成两段虚拟文本

    \subsection{图片示例}
    \begin{figure}[H]
        \centering
        \includegraphics[width=0.9\linewidth]{example-image-a}
        \caption{第一栏中的示例图片}
    \end{figure}

    \columnbreak % 手动分栏

    \section{第二栏内容}
    \subsection{定理环境}
    \begin{verbatim}
    \begin{theorem}
    任意大于2的偶数可以表示为两个素数之和
    \end{theorem}
    \end{verbatim}

    \subsection{代码示例}
    \begin{verbatim}
    def fibonacci(n):
        a, b = 0, 1
        for _ in range(n):
            yield a
            a, b = b, a + b
    \end{verbatim}

    \subsection{颜色文本}
    {\color{red}红色文本}、{\color{blue}蓝色文本}和{\color{green}绿色文本}示例

    \subsection{引用文献}
    以下是文献引用示例:
    \begin{enumerate}
        \item 文献 \cite{key1}
        \item 文献\cite{key2}
        \item 文献\cite{key3}
    \end{enumerate}

    \lipsum[3-5] % 生成更多虚拟文本

\end{multicols}

\section{代码块}

\begin{lstlisting}
666
\end{lstlisting}

\end{document}
